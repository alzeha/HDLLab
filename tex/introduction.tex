\chapter*{Introduction}

The onset of digital age has caused a dramatic change in the conception of electronic devices. Digital devices have gradually replaced most analog devices, so that the latter are now only used for specific applications. Thus, it has become necessary to develop converters (such as Analog-Digital and Digital-Analog Converters), in order to have digital devices communicating to the outside analogic environment.\\
Concurrently, the development of digital signal processing has been increasing, in order to meet the demand in the field. Thus, new architectures of digital filters for subsampling, noise reduction, and so on keep emerging. One of the advantages of these filters in comparison to analog filters is the fact that they are not subject to non-linearity behavior, which considerably reduces their complexity.\\
\\
The aim of this lab is to digitally filter an analog waveform. To this end, several analog and digital components have to be implemented, such as a DAC, an ADC and a digital filter. The performance of these designs has to be evaluated with provided WAV signals. Therefore, several architectures can be compared and optimized in order to achieve better results. For this purpose, several design tools are provided, such as ModelSim for Verilog simulation, Cadence Virtuoso Analog Design Environment for mixed signals simulation and Synopsis Design Vision for Verilog module synthesis.\\
\\
This report is organized as follows. In a first part, an implementation of basic components provided as tutorial tasks is briefly described. Then, the design and performance evaluation of the developed ADC and digital filter in this lab are detailed in the second and third part respectively. Lastly, the full downsampling circuit is evaluated in the fourth part.
